\documentclass[11pt]{article}
\usepackage{amsmath,amssymb,amsmath,amsthm,amsfonts}
\usepackage{latexsym,graphicx}
\usepackage{fullpage,color}
\usepackage{url}
\usepackage[pdftex,bookmarks,colorlinks=true,citecolor=blue]{hyperref}
\usepackage{natbib}
\usepackage{graphicx,subfigure}
\usepackage{algorithm}
\usepackage{algorithmic}
\usepackage{listings}
\usepackage{xcolor}
\usepackage{framed}
\usepackage{color}

% \colorlet{shadecolor}{orange!15}
\colorlet{NextBlue}{orange!15!green!50!blue!75}

\numberwithin{equation}{section}

\pagestyle{plain}

\setlength{\oddsidemargin}{0in}
\setlength{\topmargin}{0in}
\setlength{\textwidth}{6.5in}
\setlength{\textheight}{8.5in}

\newtheorem{fact}{Fact}[section]
\newtheorem{question}{Question}[section]
\newtheorem{lemma}{Lemma}[section]
\newtheorem{theorem}[lemma]{Theorem}
\newtheorem{assumption}[lemma]{Assumption}
\newtheorem{corollary}[lemma]{Corollary}
\newtheorem{prop}[lemma]{Proposition}
\newtheorem{claim}{Claim}[section]
\newtheorem{remark}{Remark}[section]
\newtheorem{definition}{Definition}[section]
\newtheorem{prob}{Problem}[section]
\newtheorem{conjecture}{Conjecture}[section]
\newtheorem{property}{Property}[section]

\def\A{{\bf A}}
\def\a{{\bf a}}
\def\B{{\bf B}}
\def\bb{{\bf b}}
\def\C{{\bf C}}
\def\c{{\bf c}}
\def\D{{\bf D}}
\def\d{{\bf d}}
\def\E{{\bf E}}
\def\e{{\bf e}}
\def\F{{\bf F}}
\def\f{{\bf f}}
\def\g{{\bf g}}
\def\h{{\bf h}}
\def\G{{\bf G}}
\def\H{{\bf H}}
\def\I{{\bf I}}
\def\K{{\bf K}}
\def\k{{\bf k}}
\def\LL{{\bf L}}
\def\M{{\bf M}}
\def\m{{\bf m}}
\def\N{{\bf N}}
\def\n{{\bf n}}
\def\PP{{\bf P}}
\def\pp{{\bf p}}
\def\Q{{\bf Q}}
\def\q{{\bf q}}
\def\R{{\bf R}}
\def\rr{{\bf r}}
\def\S{{\bf S}}
\def\s{{\bf s}}
\def\T{{\bf T}}
\def\tt{{\bf t}}
\def\U{{\bf U}}
\def\u{{\bf u}}
\def\V{{\bf V}}
\def\v{{\bf v}}
\def\W{{\bf W}}
\def\w{{\bf w}}
\def\X{{\bf X}}
\def\x{{\bf x}}
\def\Y{{\bf Y}}
\def\y{{\bf y}}
\def\Z{{\bf Z}}
\def\z{{\bf z}}
\def\0{{\bf 0}}
\def\1{{\bf 1}}



\def\AM{{\mathcal A}}
\def\CM{{\mathcal C}}
\def\DM{{\mathcal D}}
\def\EM{{\mathcal E}}
\def\GM{{\mathcal G}}
\def\FM{{\mathcal F}}
\def\IM{{\mathcal I}}
\def\JM{{\mathcal J}}
\def\KM{{\mathcal K}}
\def\LM{{\mathcal L}}
\def\NM{{\mathcal N}}
\def\OM{{\mathcal O}}
\def\PM{{\mathcal P}}
\def\SM{{\mathcal S}}
\def\TM{{\mathcal T}}
\def\UM{{\mathcal U}}
\def\VM{{\mathcal V}}
\def\WM{{\mathcal W}}
\def\XM{{\mathcal X}}
\def\YM{{\mathcal Y}}
\def\RB{{\mathbb R}}
\def\RBmn{{\RB^{m\times n}}}
\def\EB{{\mathbb E}}
\def\PB{{\mathbb P}}

\def\TX{\tilde{\bf X}}
\def\TA{\tilde{\bf A}}
\def\tx{\tilde{\bf x}}
\def\ty{\tilde{\bf y}}
\def\TZ{\tilde{\bf Z}}
\def\tz{\tilde{\bf z}}
\def\hd{\hat{d}}
\def\HD{\hat{\bf D}}
\def\hx{\hat{\bf x}}
\def\nysA{{\tilde{\A}_c^{\textrm{nys}}}}

\def\alp{\mbox{\boldmath$\alpha$\unboldmath}}
\def\bet{\mbox{\boldmath$\beta$\unboldmath}}
\def\epsi{\mbox{\boldmath$\epsilon$\unboldmath}}
\def\etab{\mbox{\boldmath$\eta$\unboldmath}}
\def\ph{\mbox{\boldmath$\phi$\unboldmath}}
\def\pii{\mbox{\boldmath$\pi$\unboldmath}}
\def\Ph{\mbox{\boldmath$\Phi$\unboldmath}}
\def\Ps{\mbox{\boldmath$\Psi$\unboldmath}}
\def\ps{\mbox{\boldmath$\psi$\unboldmath}}
\def\tha{\mbox{\boldmath$\theta$\unboldmath}}
\def\Tha{\mbox{\boldmath$\Theta$\unboldmath}}
\def\muu{\mbox{\boldmath$\mu$\unboldmath}}
\def\Si{\mbox{\boldmath$\Sigma$\unboldmath}}
\def\si{\mbox{\boldmath$\sigma$\unboldmath}}
\def\Gam{\mbox{\boldmath$\Gamma$\unboldmath}}
\def\Lam{\mbox{\boldmath$\Lambda$\unboldmath}}
\def\De{\mbox{\boldmath$\Delta$\unboldmath}}
\def\Ome{\mbox{\boldmath$\Omega$\unboldmath}}
\def\Pii{\mbox{\boldmath$\Pi$\unboldmath}}
\def\varepsi{\mbox{\boldmath$\varepsilon$\unboldmath}}
\newcommand{\ti}[1]{\tilde{#1}}
\def\Ncal{\mathcal{N}}
\def\argmax{\mathop{\rm argmax}}
\def\argmin{\mathop{\rm argmin}}

\def\ALG{{\AM_{\textrm{col}}}}

\def\mean{\mathsf{mean}}
\def\std{\mathsf{std}}
\def\bias{\mathsf{bias}}
\def\var{\mathsf{var}}
\def\sgn{\mathsf{sgn}}
\def\tr{\mathsf{tr}}
\def\rk{\mathrm{rank}}
\def\nnz{\mathsf{nnz}}
\def\poly{\mathrm{poly}}
\def\diag{\mathsf{diag}}
\def\Diag{\mathsf{Diag}}
\def\const{\mathrm{Const}}
\def\st{\mathsf{s.t.}}
\def\vect{\mathsf{vec}}
\def\sech{\mathrm{sech}}
\def\sigmoid{\mathsf{sigmoid}}

\newcommand{\red}[1]{{\color{red}#1}}



\def\argmax{\mathop{\rm argmax}}
\def\argmin{\mathop{\rm argmin}}

\newenvironment{note}[1]{\medskip\noindent \textbf{#1:}}%
        {\medskip}


\newcommand{\etal}{{\em et al.}\ }
\newcommand{\assign}{\leftarrow}
\newcommand{\eps}{\epsilon}

\newcommand{\opt}{\textrm{\sc OPT}}
\newcommand{\script}[1]{\mathcal{#1}}
\newcommand{\ceil}[1]{\lceil #1 \rceil}
\newcommand{\floor}[1]{\lfloor #1 \rfloor}



\lstset{ %
extendedchars=false,            % Shutdown no-ASCII compatible
language=Python,                % choose the language of the code
xleftmargin=1em,
xrightmargin=1em,
basicstyle=\footnotesize,    % the size of the fonts that are used for the code
tabsize=3,                            % sets default tabsize to 3 spaces
numbers=left,                   % where to put the line-numbers
numberstyle=\tiny,              % the size of the fonts that are used for the line-numbers
stepnumber=1,                   % the step between two line-numbers. If it's 1 each line
                                % will be numbered
numbersep=5pt,                  % how far the line-numbers are from the code   %
keywordstyle=\color[rgb]{0,0,1},                % keywords
commentstyle=\color[rgb]{0.133,0.545,0.133},    % comments
stringstyle=\color[rgb]{0.627,0.126,0.941},      % strings
backgroundcolor=\color{white}, % choose the background color. You must add \usepackage{color}
showspaces=false,               % show spaces adding particular underscores
showstringspaces=false,         % underline spaces within strings
showtabs=false,                 % show tabs within strings adding particular underscores
frame=single,                 % adds a frame around the code
%captionpos=b,                   % sets the caption-position to bottom
breaklines=true,                % sets automatic line breaking
breakatwhitespace=false,        % sets if automatic breaks should only happen at whitespace
%title=\lstname,                 % show the filename of files included with \lstinputlisting;
%                                % also try caption instead of title
mathescape=true,escapechar=?    % escape to latex with ?..?
escapeinside={\%*}{*)},         % if you want to add a comment within your code
%columns=fixed,                  % nice spacing
%morestring=[m]',                % strings
%morekeywords={%,...},%          % if you want to add more keywords to the set
%    break,case,catch,continue,elseif,else,end,for,function,global,%
%    if,otherwise,persistent,return,switch,try,while,...},%
}


\begin{document}

%\setlength{\fboxrule}{.5mm}\setlength{\fboxsep}{1.2mm}
%\newlength{\boxlength}\setlength{\boxlength}{\textwidth}
%\addtolength{\boxlength}{-4mm}


\title{Privacy attacks}

\author{\textbf{Mengjiao Zhang} \\ Stevens Institute of technology}

%\date{ }

\maketitle

% \begin{abstract}
% % This lecture note describes synchronous parallel accelerated gradient descent (AGD) for empirical risk minimization ERM.
% % We first describe AGD for solving ERM.
% % We then show how to parallelize AGD; in particular, we assume there is a central parameter server and the data are partitioned among the worker nodes.
% % We finally use Python to write a simulator that mimics synchronous parallel AGD.


% \end{abstract}




\section{Privacy attacks} 

Evasion and poisoning attacks: make the model output an incorrect predicition;

Privacy attacks: to exact sensitive information from the model, such as the underlying data or the model itself.

\section{The data}

Want the model to spot certain patterns in the training data but can also generalize its "learning" to other unseen datasets. The \textbf{generalization} is important.

\colorbox{orange!15}{Membership inference attack}

When the attacker has a data point and wants to know if it belonged to the original dataset.

\colorbox{orange!15}{Data extraction or model inversion}

Given just BlackBox access to a classifier, to reconstruct some of the underlying data. Such attacks try to extract an average representation of each of the classes the model was trained on.

\section{The Model}

Reason to steal it(perform "model extraction"):
\begin{enumerate}
    \item the model is valuable
    \item Want to launch a WhiteBox evasion attack
\end{enumerate}

\section{Defenses}

\begin{enumerate}
    \item API hardening
    \begin{itemize}
        \item Not expose the API at all
        \item Expose only hard labels and not the confidence scores
        \item If you have to expose confidence scores, you could reduce the dimensionality. Eg instead of showing 64\% you could bucket the scores into 3 or 4 buckets and just show “high”
    \end{itemize}
    \item data sanitization\\
    filter the sensitive information
    \item model hardening
    \begin{itemize}
        \item Model choice — eg Bayesian models are more robust than Decision Trees
        \item Fit control — in general overftting makes it easier to extract data out of your model, so using regularization is a good idea
        \item Knowledge control — keeping the model and its development restricted to a small group of people will help prevent leaking knowledge useful to the attacker
    \end{itemize}
    \item detection\\
    pattern of queries are likely to be anomalous compared to what your actual users would do
    \item differential privacy\\
    it aims to prove that two models differing by exactly 1 sample will produce similar predictions.
\end{enumerate}

differential privacy: injecting \textbf{noise} (or \textbf{"randomness"}) into the model. The ways could do it:

\begin{itemize}
    \item Perturb user’s input into a common training pool (eg when a user sends data to a server x\% is replaced with random numbers)
    \item Perturb the underlying data
    \item Perturb the parameters of the model (eg inject noise into the parameter update process)
    \item Perturb the loss function (similar to eg $L_2$ regularization)
    \item Perturb output during prediction (eg to x\% of users your API sends random noise)
\end{itemize}


\bibliographystyle{plain}

%\markboth{\bibname}{\bibname}
% \bibliography{bib/decentralized,bib/distributed,bib/system}
\bibliography{bib/decentralized,bib/distributed,bib/system,bib/poison}

\end{document}
